Tato kapitola se zaměřuje na návrh a implementaci softwarové aplikace v jazyce Python, která podporuje výpočty kompromisů pomocí metody BeCoMe. Cílem je vytvořit praktický nástroj, který převádí teoretické principy, popsané v předchozí kapitole, do funkčního softwarového řešení. Kapitola popisuje jednotlivé kroky vývoje od návrhu architektury, přes implementaci klíčových funkcí, až po testování a zhodnocení výsledné aplikace.

\section{Návrh architektury aplikace}
Základem pro implementaci je robustní objektově orientovaný návrh, který zajišťuje modularitu a rozšiřitelnost aplikace. V souladu s principy popsanými v podkapitole 3.3 byl navržen systém tříd, který logicky odděluje datovou reprezentaci od výpočetní logiky. Architektura je znázorněna pomocí UML diagramu tříd na obrázku~\ref{fig:uml_class_diagram}.

\begin{figure}[H]
    \centering
    \caption{UML diagram základních tříd pro implementaci metody BeCoMe}\label{fig:uml_class_diagram}
    \includegraphics[width=0.7\textwidth]{../figures/uml_class_diagram.png}
    \source{Zdroj: vlastní zpracování}
\end{figure}

Diagram na obrázku~\ref{fig:uml_class_diagram} znázorňuje tři klíčové třídy navrhované aplikace:
\begin{itemize}
    \item \textbf{Třída \texttt{FuzzyTriangleNumber}} zapouzdřuje data pro jedno trojúhelníkové fuzzy číslo, včetně atributů \textit{lower\_bound}, \textit{peak} a \textit{upper\_bound} a metody \texttt{get\_centroid()}.
    \item \textbf{Třída \texttt{ExpertOpinion}} reprezentuje jedno expertní stanovisko a pomocí kompozice (vyznačeno plným diamantem) obsahuje právě jednu instanci třídy \texttt{FuzzyTriangleNumber}.
    \item \textbf{Třída \texttt{BeCoMeCalculator}} představuje hlavní výpočetní logiku. Prostřednictvím agregace (vyznačeno prázdným diamantem) pracuje se seznamem jednoho či více (\enquote{1..*}) expertních stanovisek. Tento vztah naznačuje, že instance \texttt{ExpertOpinion} mohou existovat i nezávisle na samotném kalkulátoru.
\end{itemize}
Tento návrh vizualizuje principy modularity a zapouzdření, kde každá třída má jasně definovanou zodpovědnost, což usnadňuje následnou implementaci, testování a budoucí rozšiřování.

\begin{comment}
\section{Implementace a uživatelské rozhraní}
Na základě navržené architektury byla vytvořena aplikace v jazyce Python. V této podkapitole je popsána zvolená forma uživatelského rozhraní a klíčové aspekty implementace výpočetní logiky.

\subsection{Uživatelské rozhraní v prostředí Jupyter Notebook}

\subsection{Implementace výpočetní logiky}

\section{Testování a zhodnocení aplikace}

\subsection{Testování a ověření funkčnosti}

\subsection{Zhodnocení přínosu pro uživatele}
\end{comment}