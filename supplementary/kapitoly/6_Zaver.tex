\begin{comment}
V závěru bakalářské práce shrnujeme hlavní poznatky a výsledky dosažené během vývoje softwarové aplikace pro podporu výpočtu kompromisů metodou BeCoMe. Tato práce se zaměřila na implementaci teoretických principů metody BeCoMe do funkčního softwarového řešení v jazyce Python, které umožňuje efektivní zpracování expertních odhadů a nalezení optimálního kompromisního řešení.

Během vývoje aplikace jsme se setkali s různými výzvami, jako je zpracování fuzzy čísel a implementace uživatelského rozhraní. Tyto výzvy byly překonány díky důkladné analýze odborné literatury a aplikaci osvědčených metodik. Výsledná aplikace nejenže splňuje stanovené cíle, ale také nabízí možnosti pro budoucí rozšíření a vylepšení.

Získané výsledky ukazují, že metoda BeCoMe je efektivním nástrojem pro skupinové rozhodování v podmínkách neurčitosti a může být aplikována v různých oblastech, jako je krizové řízení nebo environmentální management. V budoucnu by bylo zajímavé prozkoumat možnosti integrace této metody s dalšími rozhodovacími nástroji a technikami, což by mohlo přispět k ještě širšímu využití v praxi.

Celkově lze říci, že tato bakalářská práce přispěla k rozvoji znalostí v oblasti skupinového rozhodování a poskytla praktické řešení pro aplikaci metody BeCoMe v reálných situacích.
\end{comment}