\begin{comment}
Tato kapitola se zaměřuje na analýzu výsledků získaných z implementace softwarové aplikace pro výpočet kompromisů metodou BeCoMe. V první části se budeme věnovat prezentaci výsledků, které byly získány na základě testovacích scénářů, a v druhé části se zaměříme na diskusi o těchto výsledcích, jejich významu a možných implikacích.

\section{Prezentace výsledků}
V rámci testování aplikace byly provedeny různé scénáře, které simulovaly reálné situace rozhodování. Byly shromážděny expertní odhady a aplikace provedla výpočty na základě těchto dat. Výsledky ukazují, že metoda BeCoMe dokáže efektivně agregovat názory expertů a nalézt optimální kompromis.

\subsection{Diskuse}
Výsledky naznačují, že použití metody BeCoMe přináší výhody v situacích, kdy jsou názory expertů rozdělené. Aplikace umožňuje uživatelům lépe porozumět rozložení názorů a identifikovat nejlepší možný kompromis. Nicméně, je důležité vzít v úvahu, že výsledky mohou být ovlivněny kvalitou a rozmanitostí expertních odhadů. Dále je třeba zvážit možnosti budoucího rozšíření aplikace, které by mohly zahrnovat pokročilejší analytické nástroje a vizualizace.

Tato kapitola tedy poskytuje důležité poznatky o účinnosti metody BeCoMe a naznačuje směry pro další výzkum a vývoj.
\end{comment}