Rozhodování v komplexním a neurčitém prostředí, ať už jde o krizové řízení či státní finance, se stále častěji musí opírat o úsudek expertů. Analytické modely zde často selhávají. Problém však nastává, když jsou názory odborníků nesourodé nebo dokonce protichůdné – tehdy ztrácejí běžné agregační metody jako aritmetický průměr či medián svou vypovídací hodnotu.

Právě pro tyto situace byla na Provozně ekonomické fakultě ČZU v Praze vyvinuta metoda BeCoMe (\textit{Best Compromise Mean}). Její síla, založená na principech fuzzy logiky, spočívá ve schopnosti nalézt teoreticky nejlepší možný kompromis i mezi výrazně odlišnými stanovisky. Současná referenční implementace v podobě tabulkového procesoru skvěle demonstruje jednoduchost a transparentnost výpočtu, její praktické nasazení pro složitější úlohy či integraci do jiných systémů však odhaluje prostor pro další vývoj. Mezi hlavní výzvy patří závislost na komerční platformě, omezená rozšiřitelnost a obtížnější automatizace.

Překonání těchto limitací je hlavním motivem této bakalářské práce. Mým cílem je navrhnout a implementovat moderní softwarovou aplikaci v jazyce Python, která bude nejen platformně nezávislá, ale také snadno rozšiřitelná. Výsledkem by měl být otevřený nástroj, který zefektivní zpracování expertních stanovisek a usnadní hledání optimálních rozhodnutí.

K dosažení tohoto cíle je práce logicky strukturována. V úvodních kapitolách vymezím metodiku a představím teoretický rámec, zahrnující jak samotnou metodu BeCoMe, tak klíčové softwarově-inženýrské principy navrženého řešení. Stěžejní část práce je pak věnována detailnímu popisu praktické implementace backendové a frontendové služby a jejich kontejnerizaci. Závěrečná diskuse zhodnotí přínos vytvořené aplikace a porovná její vlastnosti s původním řešením.