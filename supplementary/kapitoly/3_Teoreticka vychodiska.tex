Tato kapitola se věnuje teoretickému vymezení problematiky, na které je založena tato bakalářská práce. Cílem je představit kontext skupinového expertního rozhodování a vysvětlit základní principy teorie fuzzy množin. Současně popisuje klíčové principy softwarového inženýrství, jako je objektově orientovaný přístup a vizuální modelování pomocí jazyka UML, které tvoří metodický rámec pro následnou implementaci. V závěru kapitoly je detailně rozebrána metoda BeCoMe, jejíž softwarová implementace je jádrem této práce. Rešerše vychází z klíčového vědeckého článku od \textcite{Vrana2021}, zakládajícího článku o fuzzy množinách od \textcite{Zadeh1965} a související odborné literatury jako \textcite{Klir1995}. Kontext rozhodovacích metod je čerpán z \textcite{Subrt2011} a doplněn o informace z rozhovoru s docentem Tyrychtrem (in \textcite{Prokopova2023}). Teoretický základ pro softwarové inženýrství a objektové modelování je podložen prací autorů z Provozně ekonomické fakulty ČZU \autocite{Merunka2005}.

\section{Rozhodování v podmínkách neurčitosti}
Rozhodovací procesy v reálném světě, ať už v oblasti environmentálního managementu, krizového řízení nebo ekonomiky, jsou často charakteristické vysokou mírou komplexity, neurčitosti a neúplnosti dostupných dat. V takových situacích často neexistují spolehlivé analytické či statistické modely, které by dokázaly s jistotou predikovat důsledky jednotlivých variant rozhodnutí. Proto je běžnou a osvědčenou praxí spoléhat se na úsudek a intuici skupiny expertů z dané oblasti \autocite{Vrana2021}.

Zapojení více expertů má za cíl eliminovat subjektivní chyby a šum v úsudku jednotlivce a dospět k robustnějšímu, většinovému názoru. Jak však upozorňuje \textcite{Klir1995}, v rámci multipersonálního rozhodování (\textit{multiperson decision making}) často dochází k situaci, kdy jsou názory expertů či jiných zúčastněných stran (tzv.\ stakeholderů) nesourodé, nebo dokonce protichůdné, jak na to upozorňuje Tyrychtr (\textit{in} \textcite{Prokopova2023}). Běžné agregační operátory, jako je aritmetický průměr nebo medián, v takových případech selhávají. Mohou vést k výsledkům, které nereprezentují názor žádné ze skupin a ve skutečnosti mohou být pro všechny nepřijatelné. Cílem tedy není najít průměr za každou cenu, ale nalézt takové řešení, které představuje nejlepší možný kompromis přijatelný pro všechny zúčastněné strany. 

Myšlenka využití fuzzy množin pro modelování rozhodovacích procesů, kde cíle a omezení nejsou ostře definovány, byla poprvé komplexně formulována v přelomové práci \textcite{Bellman1970}. Jejich model, založený na symetrickém přístupu k cílům a omezením jako k fuzzy množinám, položil základy pro celou oblast fuzzy rozhodování a ukázal cestu k hledání optimálních kompromisů v neurčitém prostředí.

Zatímco přístup Bellmana a Zadeha využívá pro modelování neurčitosti fuzzy množiny, je vhodné zmínit, že klasická teorie rozhodování nabízí pro situace s~úplnou nejistotou (kde nejsou známy pravděpodobnosti jednotlivých stavů) několik jiných kritérií. Jak popisuje Brožová (in \textcite{Subrt2011}), mezi tyto přístupy patří například pesimistické Waldovo maximinové kritérium, optimistické maximaxové kritérium nebo Savageovo kritérium minimaxové ztráty. Tyto metody se však opírají o~matici výsledků pro diskrétní stavy a~nezpracovávají neurčitost plynoucí z~vágnosti expertních odhadů takovým způsobem jako teorie fuzzy množin, která je základem pro metodu BeCoMe.

\section{Teorie fuzzy množin}
Jedním z klíčových problémů expertního rozhodování je vágnost a neurčitost lidského úsudku. Jak uvádí \textcite{Zadeh1965} ve svém zakládajícím článku, mnoho tříd objektů v reálném světě nemá přesně definovaná kritéria příslušnosti. Pojmy jako \enquote{třída zvířat}, \enquote{třída krásných žen} nebo \enquote{čísla mnohem větší než 1} jsou z podstaty imprecizní a hrají klíčovou roli v lidském myšlení. Tradiční matematická teorie množin, založená na ostré dichotomii, není pro modelování takových konceptů vhodná. Zavedení konceptu fuzzy množiny je v odborné literatuře často považováno za \textit{paradigmatický posun} \autocite{Klir1995}, který umožnil opustit striktní rámec klasické logiky a matematicky modelovat neurčitost a vágnost, jež jsou přirozenou součástí lidského úsudku.

Právě pro formalizaci této imprecize představil \textcite{Zadeh1965} koncept fuzzy množiny (\textit{fuzzy set}). Na rozdíl od klasické (tzv. \textit{crisp}) množiny je fuzzy množina charakterizována funkcí příslušnosti (\textit{membership function}), která každému objektu z univerza přiřazuje stupeň příslušnosti – reálné číslo v intervalu $[0, 1]$. Hodnota blízká 1 značí vysoký stupeň příslušnosti, zatímco hodnota blízká 0 značí nízký. Tento přístup umožňuje matematicky modelovat plynulé přechody a expertní nejistotu, která není statistické povahy, ale pramení z vágnosti samotných pojmů. Dále definoval základní operace pro fuzzy množiny, jako je sjednocení a průnik, které tvoří základ pro složitější operace.

Speciálním případem fuzzy množin jsou fuzzy čísla, která reprezentují lingvistické koncepty jako \enquote{přibližně deset} nebo \enquote{vysoká hodnota}. Tyto koncepty jsou v teorii fuzzy množin označovány jako lingvistické proměnné (\textit{linguistic variables}), jejichž hodnotami jsou právě fuzzy čísla \autocite{Klir1995}. Metoda BeCoMe tak ve své podstatě pracuje s agregací expertních odhadů vyjádřených prostřednictvím těchto proměnných. V kontextu metody BeCoMe jsou expertní odhady nejčastěji reprezentovány pomocí trojúhelníkových fuzzy čísel, která jsou definována třemi hodnotami: dolní mezí, nejpravděpodobnější hodnotou (jádrem) a horní mezí \autocite{Vrana2021}. Grafické znázornění takového čísla pomocí trojúhelníkové funkce příslušnosti, kde osa X reprezentuje univerzum diskurzu a osa Y stupeň příslušnosti, je uvedeno na obrázku~\ref{fig:triangular_membership}.

\begin{figure}[H]
    \centering
    \caption{Příklad trojúhelníkové funkce příslušnosti}\label{fig:triangular_membership}
    \includegraphics[width=0.6\textwidth]{../figures/triangular_membership.png}
    \source{Zdroj: \textcite[Figure 1.2]{Klir1995}}
\end{figure}

\section{Objektově orientovaný přístup k vývoji software}
Tvorba softwarové aplikace, která je cílem této práce, vyžaduje nejen znalost teoretických základů metody, ale také systematický přístup k návrhu a implementaci. Jak uvádějí \textcite{Lott2021}, moderní vývoj software kombinuje fáze analýzy, návrhu a programování, přičemž klíčovou roli hraje zvolené programovací paradigma. Objektově orientovaný přístup (OOP) vznikl jako snaha o překonání tzv. sémantické mezery mezi způsobem, jakým o~reálném světě uvažuje člověk, a~způsobem, jakým funguje počítač \autocite{Merunka2005}. Tento přístup umožňuje modelovat komplexní systémy pomocí interagujících objektů, které zapouzdřují data (atributy) a chování (metody).

Pro implementaci byl zvolen programovací jazyk Python. Jedná se o~multi-paradigmatický jazyk, což znamená, že plně podporuje objektově orientované, procedurální i~funkcionální programování. Ačkoliv programátora striktně nenutí používat objekty pro všechny aspekty programu, poskytuje plnohodnotnou podporu pro všechny klíčové principy OOP \autocite{Merunka2005}:
\begin{itemize}
    \item \textbf{Zapouzdření (Encapsulation):} Schopnost sdružovat data a metody, které s nimi pracují, do jedné entity -- objektu. V této práci je tento princip demonstrován na třídě \texttt{FuzzyTriangleNumber}, která zapouzdřuje hodnoty \texttt{lower\_bound}, \texttt{peak} a \texttt{upper\_bound} spolu s~metodou \texttt{get\_centroid()}.
    \item \textbf{Dědičnost (Inheritance):} Umožňuje vytvářet nové třídy na základě již existujících, čímž se podporuje znovupoužitelnost kódu. V kontextu této práce by bylo možné například vytvořit různé typy expertních stanovisek (číselné, lingvistické) jako podtřídy obecné třídy \texttt{Opinion}.
    \item \textbf{Polymorfismus (Polymorphism):} Schopnost objektů různých tříd reagovat na stejnou zprávu (volání metody) specifickým způsobem. Python tento princip silně podporuje díky dynamickému typování.
\end{itemize}

Zvolení OOP pro tuto aplikaci bylo motivováno jeho klíčovými přínosy, jako je modularita, znovupoužitelnost a lepší udržovatelnost kódu \autocite{Merunka2005}. Tyto vlastnosti jsou zásadní pro vytvoření robustní a flexibilní aplikace, jakou je implementace metody BeCoMe.

Při návrhu struktury aplikace je vhodné využít vizuální modelovací jazyky, jako je UML (\textit{Unified Modeling Language}). Jak uvádí \textcite{Merunka2005}, UML je univerzální standard pro vizuální modelování systémů, který umožňuje graficky znázornit jednotlivé třídy, jejich atributy, metody a vztahy mezi nimi. Tento vizuální návrh slouží jako plán pro samotnou implementaci a usnadňuje komunikaci o architektuře systému \autocite{Lott2021}. Příklad obecného diagramu tříd je uveden na obrázku~\ref{fig:uml_example}. Konkrétní návrh tříd pro implementaci metody BeCoMe je pak detailněji rozpracován v praktické části práce na obrázku~\ref{fig:uml_class_diagram}.

\begin{figure}[H]
    \centering
    \caption{Příklad obecného UML diagramu tříd s atributy a metodami}\label{fig:uml_example}
    \includegraphics[width=0.8\textwidth]{../figures/uml_example.png}
    \source{Zdroj: \textcite[Figure 1.5]{Lott2021}}
\end{figure}

Diagram na obrázku~\ref{fig:uml_example} ilustruje základní stavební prvky diagramu tříd. Každá třída je reprezentována obdélníkem, který je rozdělen na tři části: název třídy (např. \texttt{Orange}), její atributy a její metody. Před názvy atributů a metod je uveden symbol viditelnosti, například \enquote{+} pro veřejné (\textit{public}) členy. Atributy (např. \texttt{+weight: float}) definují data, která objekt uchovává, zatímco metody (např. \texttt{+squeeze(): float}) definují chování, které objekt může vykonávat. Vztah mezi třídami je znázorněn spojnicí. V tomto případě se jedná o asociaci, která ukazuje, že instance tříd \texttt{Orange} a \texttt{Basket} spolu souvisí. Čísla a symboly na koncích spojnice (\texttt{*} a \texttt{1}) definují násobnost (\textit{multiplicity}) vztahu, která specifikuje, kolik instancí jedné třídy může být spojeno s jednou instancí druhé třídy.

Kromě obecné asociace (znázorněné plnou čarou) definuje UML i silnější formy vztahů, jako je agregace a kompozice, které vyjadřují vztah celku a jeho částí. \textcite{Merunka2005} popisují agregaci jako vztah, kde součásti mohou existovat i nezávisle na celku, zatímco u kompozice je existence součástí přímo vázána na existenci celku. Tyto vztahy jsou klíčové pro modelování složitějších datových struktur, jak je ukázáno v návrhu architektury této práce (obrázek~\ref{fig:uml_class_diagram}).

\section{Architektonické vzory moderních webových aplikací}
Kromě objektově orientovaného přístupu, který se zaměřuje na vnitřní strukturu kódu, je pro vývoj moderních aplikací klíčový i~výběr správného architektonického vzoru na úrovni celého systému. Pro potřeby této práce jsou relevantní zejména následující koncepty.

\subsection{Mikroservisní architektura}
Tradiční monolitické aplikace, kde je veškerá logika spojena do jednoho celku, narážejí při rostoucí složitosti na své limity v~oblasti škálovatelnosti, údržby a~rychlosti nasazování nových funkcí. Jako odpověď na tyto výzvy se objevil architektonický vzor mikroservisní architektury. Tento model, jak popisuje \textcite{Newman2021}, strukturuje aplikaci jako soubor malých, nezávislých služeb, které jsou modelovány kolem obchodních domén. Každá služba je vyvíjena a~nasazována nezávisle, což umožňuje týmům pracovat autonomně a~výrazně zrychluje inovační cyklus. Právě tento princip \textit{nezávislé nasaditelnosti} (\textit{independent deployability}) je hlavním motivem pro volbu této architektury v~rámci této práce.

Obrázek~\ref{fig:monolith_change} názorně ilustruje klíčový problém tradiční vrstvené architektury, jak jej popisuje Newman. Ačkoliv je požadavek na změnu z~pohledu uživatele jednoduchý (přidání možnosti volby oblíbeného žánru), jeho implementace, znázorněná jako \enquote{Scope of change}, vyžaduje zásah do všech tří vrstev systému. To v~praxi znamená nutnost koordinace práce mezi třemi různými týmy (Frontend team, Backend team, správci databází (DBAs)), což vede k~vyšší organizační složitosti a~zvyšuje riziko zpoždění.

\begin{figure}[H]
    \centering
    \caption{Rozsah změny v tradiční vrstvené architektuře}\label{fig:monolith_change}
    \includegraphics[width=0.9\textwidth]{../figures/arch-monolith.png}
    \source{Zdroj: \textcite[Figure 1.3]{Newman2021}}
\end{figure}

Naopak, Obrázek~\ref{fig:microservice_slices} představuje řešení tohoto problému pomocí mikroservisního přístupu. Architektura zde není dělena horizontálně podle technologií, ale vertikálně podle obchodních domén (\enquote{Stock functionality}, \enquote{Purchase functionality}, \enquote{Profile functionality}). Díky tomu je stejná změna plně lokalizována v~rámci jedné služby (\enquote{Profile functionality}). Jeden autonomní tým tak může implementovat celou funkcionalitu od uživatelského rozhraní až po databázi, aniž by musel koordinovat svou práci s~ostatními. Tento přístup přímo podporuje princip nezávislé nasaditelnosti, který je hlavním motivem této práce.

\begin{figure}[H]
    \centering
    \caption{Architektura rozdělená podle obchodních domén (vertikální řezy)}\label{fig:microservice_slices}
    \includegraphics[width=0.9\textwidth]{../figures/arch-microservices.png}
    \source{Zdroj: \textcite[Figure 1.4]{Newman2021}}
\end{figure}

\subsection{Komunikace pomocí REST API}
Pro komunikaci mezi službami v~mikroservisní architektuře je nutné zvolit spolehlivý a~standardizovaný mechanismus. Nejrozšířenějším přístupem je využití REST (\textit{Representational State Transfer}) API. Tento architektonický styl, který formálně definoval \textcite{Fielding2000} ve své disertační práci, klade důraz na sadu omezujících pravidel, jako jsou \textit{stateless} komunikace (bezstavovost), oddělení klienta a~serveru a~jednotné rozhraní (\textit{uniform interface}). Dodržování těchto principů zajišťuje, že služby zůstávají volně vázané (\textit{loosely coupled}) a~mohou se vyvíjet nezávisle na sobě, což je pro mikroservisní architekturu klíčové.

Obrázek~\ref{fig:rest_diagram} detailně ilustruje principy komunikace ve stylu REST. Diagram ukazuje, jak klient (\textit{User Agent}) přistupuje k~různým zdrojům na serveru (\textit{Origin Server}) prostřednictvím standardizovaných požadavků protokolu HTTP (\textit{Hypertext Transfer Protocol}). Architektura demonstruje tři různé komunikační cesty (a, b, c):
\begin{itemize}
    \item Cesta \textbf{a} ukazuje složitý požadavek, který prochází přes několik prostředníků, jako jsou Proxy a Gateway, než se dostane na server.
    \item Cesta \textbf{b} představuje přímou komunikaci mezi klientem a~serverem, který poskytuje statické soubory.
    \item Cesta \textbf{c} ilustruje integraci se starším systémem WAIS (\textit{Wide Area Information System}) prostřednictvím Proxy serveru, což demonstruje flexibilitu REST.
\end{itemize}
Důležitým prvkem jsou také různé typy konektorů, včetně těch s~možností cachování (označené symbolem \$), které pomáhají snižovat zátěž a~zrychlovat odezvu systému.

\begin{figure}[H]
    \centering
    \caption{Základní komponenty a~komunikační cesty v~REST architektuře}\label{fig:rest_diagram}
    \includegraphics[width=0.9\textwidth]{../figures/rest_diagram.png}
    \source{Zdroj: \textcite[Figure 5.10]{Fielding2000}}
\end{figure}

\subsection{Kontejnerizace pomocí Dockeru}
Pro zajištění konzistentního a~spolehlivého běhu jednotlivých mikroslužeb v~různých prostředích – od vývojářského počítače po produkční server – se využívá technologie kontejnerizace. Jejím nejznámějším zástupcem je Docker. Kontejner, jak popisují \textcite{Kane2018}, izoluje aplikaci a~všechny její závislosti do jednoho přenositelného balíčku, který se spouští vždy stejně bez ohledu na podkladovou infrastrukturu. Tímto přístupem se řeší klasický problém \enquote{na mém stroji to funguje} a~zajišťuje se konzistence mezi vývojovým, testovacím a~produkčním prostředím, což je základním předpokladem pro spolehlivou a~rychlou dodávku softwaru.

Obrázek~\ref{fig:docker_diagram} znázorňuje základní klient-server architekturu a~pracovní postup v~ekosystému Dockeru. Vývojář na své lokální stanici (\textit{Docker client}) komunikuje s~Docker serverem, který běží na hostitelském systému a~spravuje životní cyklus kontejnerů. Hotové obrazy (\textit{images}) jsou nahrávány do centrálního úložiště (\textit{Docker registry}) a~odtud jsou stahovány zpět na server. Tento model zajišťuje, že artefakt vytvořený během vývoje je bitově identický s~artefaktem, který je následně nasazen v~produkčním prostředí, což je klíčem ke konzistenci a~spolehlivosti.

\begin{figure}[H]
    \centering
    \caption{Základní komponenty a~pracovní postup v~ekosystému Docker}\label{fig:docker_diagram}
    \includegraphics[width=0.8\textwidth]{../figures/docker_diagram.png}
    \source{Zdroj: \textcite[Figure 2.3]{Kane2018}}
\end{figure}

\section{Metoda BeCoMe}
Metoda BeCoMe spadá do širší oblasti modelů vícekriteriálního rozhodování, konkrétně do vícekriteriální analýzy variant. Cílem těchto metod je, jak uvádí Brožová (in \textcite{Subrt2011}), nalézt kompromisní řešení v~situacích, kdy jsou jednotlivé varianty hodnoceny na základě několika, často protichůdných, kritérií. Tento kontext je zásadní, protože samotný název metody BeCoMe (\textit{Best Compromise Mean}) ji explicitně řadí do tohoto paradigmatu hledání uspokojivého kompromisu, nikoli jediného, objektivně optimálního řešení.

Předchozí kapitola se věnovala principům softwarového inženýrství a objektovému návrhu. Nyní se zaměříme na samotnou metodu BeCoMe, jejíž matematický popis je základem pro implementaci. Převod těchto teoretických principů na funkční kód je procesem algoritmizace. V kontextu této práce to znamená transformovat matematické vzorce pro výpočet průměru, mediánu a kompromisu do sekvence logických kroků, které budou implementovány v rámci navržené objektové architektury a které může počítač vykonat.

Metoda BeCoMe, neboli \textit{Best Compromise Mean}, byla vyvinuta týmem vědců z Provozně ekonomické fakulty ČZU ve složení I. Vrana, J. Tyrychtr a M. Pelikán. Jejím hlavním smyslem je nalézt optimální rozhodnutí, které odpovídá nejlepšímu možnému kompromisu v názorech všech zúčastněných expertů, a to i v případě silně protichůdných stanovisek, jak zdůrazňuje Tyrychtr (\textit{in} \textcite{Prokopova2023}).

\subsection{Vznik a účel}
Metoda BeCoMe přímo navazuje na starší metodu MaxAgM (\textit{Maximum Agreement Mean}), která rovněž agreguje názory expertů se zaměřením na shodu. Metoda MaxAgM však měla několik omezení: vysokou koncepční a výpočetní složitost, obtížnou interpretaci výsledků, absenci ukazatelů přesnosti a nutnost použití specializovaného softwaru \autocite{Vrana2021}.

Jak uvádí \textcite{Vrana2021}, metoda BeCoMe byla navržena tak, aby si zachovala hlavní výhody své předchůdkyně, ale odstranila její limity. Cílem bylo vytvořit metodu, která je rychlá, univerzální, snadno implementovatelná a srozumitelná i pro uživatele bez hlubší matematické gramotnosti. Právě tato jednoduchost a absence nutnosti specializovaného softwaru (původní implementace je v~MS Excel) otevírá prostor pro vytvoření nové, flexibilnější aplikace v jazyce Python, což je cílem této práce.

\subsection{Teoretické principy}
Základní myšlenka metody BeCoMe je elegantní a intuitivní. Kompromisní řešení je definováno jako aritmetický průměr dvou klíčových statistických ukazatelů expertních stanovisek \autocite{Vrana2021}:
\begin{enumerate}
    \item Aritmetického průměru ($\Gamma$) všech expertních odhadů.
    \item Statistického mediánu ($\Omega$) všech expertních odhadů.
\end{enumerate}

Tento přístup se liší od klasického modelu rozhodování v neurčitém prostředí, jak jej navrhli \textcite{Bellman1970}. Zatímco jejich model hledá optimální rozhodnutí jako průnik fuzzy množin cílů a omezení (typicky pomocí operátoru \textit{minimum}), metoda BeCoMe nabízí modernější přístup založený na statistické agregaci expertních stanovisek, který je robustnější vůči jejich případným protichůdným názorům.

Výsledný kompromis, označovaný jako $\Gamma\Omega\text{Mean}$, je tedy dán vztahem:
\begin{equation}
\Gamma\Omega\text{Mean} = \frac{\Gamma + \Omega}{2} \label{eq:become_mean}
\end{equation}

Pokud jsou expertní odhady vyjádřeny jako trojúhelníková fuzzy čísla, výpočty se provádějí pro každou složku (dolní mez, jádro, horní mez) zvlášť. Proces výpočtu lze rozdělit do následujících kroků, jak je znázorněno na diagramu~\ref{fig:become_flowchart} \autocite[Figure 4]{Vrana2021}:
\begin{itemize}
    \item \textbf{Výpočet aritmetického průměru ($\Gamma$):} Pro soubor $M$ expertních odhadů ve formě trojúhelníkových fuzzy čísel $A_{k}C_{k}B_{k}$ (kde $k=1, \dots, M$), výsledné průměrné fuzzy číslo $\Gamma(\alpha, \gamma, \beta)$ se vypočítá jako průměr odpovídajících složek:
    \begin{equation}
    \alpha = \frac{1}{M}\sum_{k=1}^{M} A_k, \quad
    \gamma = \frac{1}{M}\sum_{k=1}^{M} C_k, \quad
    \beta = \frac{1}{M}\sum_{k=1}^{M} B_k \label{eq:arithmetic_mean}
    \end{equation}

    \item \textbf{Výpočet statistického mediánu ($\Omega$):} Nalezení mediánu je složitější, protože vyžaduje seřazení fuzzy čísel, která nemají přirozené uspořádání. Metoda BeCoMe pro tento účel zavádí jednoznačné kritérium založené na poloze těžiště (\textit{centroidu}) každého trojúhelníkového fuzzy čísla.
    \begin{enumerate}
        \item Nejprve se pro každé trojúhelníkové fuzzy číslo $A_{k}C_{k}B_{k}$ vypočítá jeho těžiště (\textit{centroid}) $G_{x_k}$ podle vzorce:
        \begin{equation}
        G_{x_k} = \frac{A_k + C_k + B_k}{3} \label{eq:centroid}
        \end{equation}
        \item Expertní odhady jsou následně seřazeny podle rostoucích hodnot jejich těžišť.
        \item Statistický medián ($\Omega$) je pak určen jako fuzzy číslo, které se v tomto seřazeném seznamu nachází uprostřed. Pokud je počet expertů $M$ lichý, medián je přímo prostřední prvek. Pokud je počet expertů sudý, medián se vypočítá jako aritmetický průměr dvou prostředních fuzzy čísel.
    \end{enumerate}

    \item \textbf{Ukazatel přesnosti ($\Delta_{\text{max}}$):} Metoda také poskytuje ukazatel přesnosti, tzv. maximální chybu, která je definována jako polovina rozdílu mezi aritmetickým průměrem a mediánem. Udává, jak moc se od sebe tyto dva ukazatele liší, a tedy jak velký je rozptyl v názorech.
\end{itemize}

\begin{figure}[H]
    \centering
    \caption{Diagram postupu metody BeCoMe}\label{fig:become_flowchart}
    \includegraphics[width=0.8\textwidth]{../figures/become_flowchart.png}
    \source{Zdroj: \textcite[Figure 4]{Vrana2021}}
\end{figure}

\subsection{Praktické využití a argumentační síla}
Metoda BeCoMe byla úspěšně aplikována v řadě reálných situací, jako je návrh protipovodňových opatření nebo rozhodování během pandemie COVID-19 pro Ústřední krizový štáb vlády ČR \autocite{Vrana2021}. Její univerzálnost umožňuje použití v jakékoliv oblasti skupinového rozhodování, od řešení státních financí po penzijní reformu. 

Klíčovou předností metody je, že poskytuje silný argument pro obhájení přijatého rozhodnutí. Tuto myšlenku přesně formuluje Tyrychtr (\textit{in} \textcite{Prokopova2023}) slovy: \textit{\enquote{Toto rozhodnutí představuje podle teorie informace nejlepší možný kompromis a při daném rozložení expertních názorů lepší kompromis neexistuje}}.

\section{Východiska pro implementaci v Pythonu}
Stávající referenční implementace metody BeCoMe je realizována pomocí tabulkového procesoru Microsoft Excel, jak je popsáno v dodatku vědeckého článku \autocite[Appendix A]{Vrana2021}. Toto řešení je plně funkční a demonstruje jednoduchost výpočtů. Má však jistá omezení:
\begin{itemize}
    \item \textbf{Závislost na platformě:} Vyžaduje komerční software MS Excel.
    \item \textbf{Omezená flexibilita:} Integrace do jiných systémů nebo automatizovaných procesů je obtížná.
    \item \textbf{Rozšiřitelnost:} Přidávání nových funkcí nebo úprava uživatelského rozhraní je neohrabaná.
\end{itemize}
Právě tato omezení definují východisko pro tuto bakalářskou práci. Cílem je vytvořit novou, nezávislou implementaci v jazyce Python, která bude platformně nezávislá, snadno rozšiřitelná a nabídne základ pro budoucí vývoj komplexnějších nástrojů pro podporu rozhodování. Tato práce se tak zaměří na převod teoretických principů popsaných v této kapitole do funkčního softwarového kódu.